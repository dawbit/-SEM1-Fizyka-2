\documentclass{article}
\usepackage[top=1in, bottom=1in, left=1in, right=1in]{geometry}
\usepackage{polski}
\usepackage[utf8]{inputenc}
\begin{document}
\title{Teoria}
\date{}
\author{}
\maketitle
\section{Wstęp teoretyczny}
\subsection{Fala akustyczna}
Fala akustyczna jest to rozchodzące się w ośrodku zaburzenie gęstości i ciśnienia w postaci fali podłużnej, któremu towarzyszą drgania cząsteczek ośrodka. Zaburzenia te polegają na przenoszeniu energii mechanicznej przez drgające cząstki ośrodka bez zmiany ich średniego położenia. Definicja fali akustycznej nie ogranicza się tylko w częstotliwościach i amplitudach słyszalności przez ludzkie zmysły.\\\\
Wychylenie z położenia równowagi cząsteczek ośrodka opisuje równanie fali:
$$y = A\sin (\omega t - kz + \phi)$$
$$\omega = \frac{2\pi}{\lambda}$$
$$k = \frac{2\pi}{T}$$
gdzie:\\

 $A$ - amplituda
 
 $k$ - wektor fali
 
 $\omega$ - częstość fali
 
 $t$ - czas
 
 $z$ - współrzędna położenia
 
 $y$ - miara odchylenia od stanu równowagi\\\\ 
Wielkościami charakteryzującymi falę akustyczną są:
\begin{itemize}
    \item Częstotliwość - określającą wysokość dźwięku
    \item Natężenie - określające głośność dźwięku
    \item Skład widmowy fali akustycznej - określający barwę dźwięku
\end{itemize}
Część okresu fali w której znajduje się punkt fali określany jest przez jej fazę i w ruchu harmonicznym wyrażana jest ona w radianach. Różnicą pomiędzy wartościami fazy dwóch okresów fali nazywamy przesunięciem fazowym.
\subsection{Prędkość dźwięku}
Prędkość dźwięku w powietrzu zależy od temperatury powietrza i ciśnienia. Im temperatura jest wyższa tym cząsteczki poruszają się szybciej przez co prędkość dźwięku jest większa.
Przy temperaturze $15$ stopnia Celsjusza, prędkość dźwięku jest równa $340,3 \frac{m}{s} \approx 1225 \frac{km}{h}$.

\subsection{Oscyloskop}
Oscyloskop jest to urządzenie pomiarowe, służące do obserwowania, obrazowania i badania przebiegów zależności pomiędzy dwiema wielkościami fizycznymi. W oscyloskopie analogowym obraz przebiegu rysowany jest na ekranie lampy oscyloskopowej, w czasie rzeczywistym, w takt zmiany przebiegu i upływu czasu. Działania lampy oscyloskopowej polega na wytworzeniu przez katodę elektronów, sformowanie ich w wąską wiązkę przez działo elektronowe oraz wysłanie jej w kierunku ekranu(co tworzy plamkę na środku ekranu). Do odchylenia wiązki służą dwie pary płytek odchylających.\\\\
Krzywe Lissajous są to krzywe parametryczne. Powstaje kiedy ciało porusza się harmoniczne w kierunki x i y. Możemy wyznaczyć je wzorem:
$$x(t) = A\sin (at + \delta)$$
$$y(t) = B\sin (bt)$$
\begin{flushright}
\begin{scriptsize}
Źródła: \textit{http://www.szkolnictwo.pl/szukaj,Fale$\_$akustyczne} \\
\textit{http://fizyka.pisz.pl/strona/145.html}\\
\textit{http://www.jeybi.prv.pl/}\\
\end{scriptsize}
\end{flushright}
\end{document}