\documentclass{article}
\usepackage[top=1in, bottom=1in, left=1in, right=1in]{geometry}
\usepackage{polski}
\usepackage[utf8]{inputenc}
\usepackage{multirow}
\usepackage{graphicx}
\usepackage{float}
\begin{document}
\title{\huge\bfseries Wyznaczanie ładunku właściwego elektronu metodą
poprzecznego pola magnetycznego (lampa Thomsona)}
\date{}
\author{}
\maketitle
\section{Wstęp teoretyczny}
\subsection{Promieniowanie beta}
\textbf{Promieniowanie beta}\footnote[1]{https://pl.wikipedia.org/wiki/Promieniowanie\_beta, z dnia: 25.05.2017} to strumień elektronów lub pozytonów, emitowany przez jądra atomowe podczas przemiany jądrowej. Jest jednym z rodzajów promieniowania jonizującego oraz
jest bardziej przenikliwe od promieniowania alfa(przenikliwe czyli zdolne do przenikania przez różne materiały).
Energia promieniowania jest zależna od rodzaju źródła, a zasięg promieniowania dodatkowo od gęstości substancji absorbującej.\\\\
\textbf{Przykładowe źródła promieniowania beta:}
\begin{itemize}
\item promieniowanie sztucznych jądrach promieniotwórczych powstających podczas reakcji jądrowych
\item rozpad izotopu sodu 22Na
\end{itemize}
\subsection{Absorpcja promieniowania beta}
\textbf{Absorpcja promieniowania beta}\footnote[2]{https://pl.wikipedia.org/wiki/Absorpcja\_promieniowania\_beta, z dnia: 25.05.2017} jest to proces pochłaniania promieniowania przez substancję. Oddziaływanie promieniowania beta z materią powoduje straty energii cząstek beta oraz zmianę toru ich ruchu.\\\\
\textbf{Zasięg masowy promieniowania}\footnote[3]{https://pl.wikipedia.org/wiki/Oslona\_przed\_promieniowaniem, z dnia: 25.05.2017} jest zależny od energii cząsteczek beta, czyli od zasięgu maksymalnego dla danego izotopu pierwiastka promieniotwórczego oraz od współczynnika pochłaniania absorbującej materii.

\end{document}