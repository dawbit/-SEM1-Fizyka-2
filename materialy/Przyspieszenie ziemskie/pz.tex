\documentclass[7pt]{article}
\usepackage[T1]{polski}
\usepackage[utf8]{inputenc}
\usepackage{amsfonts}
\usepackage[top=2cm, bottom=2cm, left=3cm, right=3cm]{geometry}

\begin{document}
\begin{center}
\LARGE {\textbf{Wyznaczanie przyspieszenia ziemskiego przy pomocy wahadła matematycznego}}
\end{center}
\begin{enumerate}
\item \large {Siła grawitacji} \\
\textit{\underline{Siła grawitacji}} to w typowym dla nas - ludzi rozumieniu przyciąganie otaczających nas przedmiotów przez glob ziemski, czyli zjawisko ciężaru. W fizyce jest to jedno z oddziaływań podstawowych, określone jako zjawisko naturalne, polegające na tym, że wszystkie obiekty , które posiadają masę oddziałują na siebie wzajemnie i przyciągają się. Kiedy znajdujemy się na powierzchni Ziemi, odległość od środka jej ciężkości jest dużo większa niż wysokość, na której możemy się przemieszczać. W takiej sytuacji można przyjąć, że pole grawitacyjne jest z dużą dokładnością jednorodne. \\
Korzystając z zależności na siłę grawitacyjną można obliczyć, że na przedmiot o masie $m$ na powierzchni naszej planety działa siła $F_g$: \\
\begin{center}
\begin{Large}
$F_g=\frac{GM_z m}{r_z^2}$
\end{Large}
\end{center}
\begin{small}
gdzie $M_z \approx 5,9736 \cdot 10^{24} kg$ - masa Ziemi, $r_z \approx 6373,14 km$, a zgodnie z drugą zasadą dynamiki:
\end{small}
\begin{Large}
\begin{center}
$a=\frac{F_g}{m}$
\end{center}
\end{Large}
\begin{flushright}
\begin{scriptsize}
Źródło: \textit{http://www.fizykon.org/grawitacja/grawitacja$\_$wprowadzenie.htm} \\
\textit{http://pl.wikipedia.org/wiki/Grawitacja}
\end{scriptsize}
\end{flushright}
\item \large {Przyspieszenie ziemskie, jednostka, zależność wartości od szerokości geograficznej i wysokości nad poziomem morza} \\
\textit{\underline{Przyspieszenie ziemskie}} - przyspieszenie grawitacyjne ciał swobodnie spadających na Ziemię, bez oporów ruchu.Podstawiając zależność na siłę można obliczyć przyspieszenie ziemskie $g$:
\begin{center}
\begin{Large}
$g=\frac{GM_z}{r^2}\approx\frac{6,6732 \cdot 10^{-11} \cdot m^3 \cdot kg^{-1} \cdot s{-2} \cdot 5,9736 \cdot 10^{24}kg}{(6373,14 km)^2} \approx 9,81 \frac{m}{s^2}$
\end{Large}
\end{center}
Jednostkami przyspieszenia ziemskiego są jednostki przyspieszenia: \\
\begin{center}
$[g]=[\gamma]=\frac{N}{kg}=\frac{m}{s^2}$
\end{center}
Wartość przyspieszenia ziemskiego zależy od szerokości geograficznej oraz wysokości nad poziomem morza. Wraz z wysokością przyspieszenie maleje odwrotnie proporcjonalnie do kwadratu odległości do środka Ziemi i jest wynikiem zmniejszania się siły grawitacji zgodnie z prawem powszechnego ciążenia. Zmniejszanie się przyspieszenia ziemskiego wraz ze zmniejszaniem szerokości geograficznej jest spowodowane działaniem pozornej siły odśrodkowej, która powstaje na skutek ruchu obrotowego Ziemi. Ponieważ siła ta jest proporcjonalna do odległości od osi obrotu, stąd największą wartość osiąga na równiku. Ponieważ siła odśrodkowa ma tu zwrot przeciwny do siły grawitacji, przyspieszenie ziemskie na równiku osiąga najmniejszą wartość. Dodatkowe zmniejszenie przyspieszenia ziemskiego w okolicach równika spowodowane jest spłaszczeniem Ziemi (większą odległością od środka Ziemi).\\
Nie obserwuje się zależności przyspieszenia ziemskiego od długości geograficznej.\\
Przybliżoną zależność przyspieszenia ziemskiego, z uwzględnieniem podanych efektów, podaje wzór:
\begin{center}
$g_{\varphi}\approx 9,78318(1+0,0053024sin^2\varphi-0,0000058sin^22\varphi)-3,086\cdot 10^{-6}h$
\end{center}
\begin{small}
gdzie: $h$ - wysokość nad poziomem morza [m], $\varphi$ - szerokość geograficzna [$^{\circ}$]
\end{small}
\begin{flushright}
\begin{scriptsize}
Źródło: \textit{http://pl.wikipedia.org/wiki/Przyspieszenie$\_$ziemskie}
\end{scriptsize}
\end{flushright}
\item \large{Wahadło matematyczne} \\
Idealne \textit{\underline{wahadło matematyczne}} to punktowa masa $m$ zawieszona na nieważkiej i nierozciągliwej lince. W rzeczywistości niewielka ciężka kulka zawieszona na mocnej lince i cienkiej nici jest dobrym przybliżeniem. 
\begin{flushright}
\begin{scriptsize}
Źródło: \textit{http://fizyka.pisz.pl/strona/121.html}
\end{scriptsize}
\end{flushright}

\item \large{Zależność okresu drgań od długości wahadła matematycznego} \\
Okres drgań wahadła matematycznego jest niezależny od masy $m$ i wynosi dla małych kątów $\alpha$
\begin{center}
$T=2\pi \sqrt{\frac{l}{g}}$
\end{center}
\begin{small}
gdzie, $l$ - długość wahadła
\end{small}
\end{enumerate}
\begin{flushright}
\begin{scriptsize}
Źródło: \textit{http://fizyka.pisz.pl/strona/121.html}
\end{scriptsize}
\end{flushright}
\end{document} 