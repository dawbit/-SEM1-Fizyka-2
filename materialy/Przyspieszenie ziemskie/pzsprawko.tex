\documentclass{article}
\usepackage[T1]{polski}
\usepackage[utf8]{inputenc}
\usepackage{graphicx}
\usepackage{enumerate}
\usepackage[top=2cm, bottom=2cm, left=2.7cm, right=2.7cm]{geometry}
\usepackage{amsfonts}
\usepackage{multirow}
\usepackage{times}

\renewcommand*\sfdefault{lmssq}
\renewcommand*\familydefault{\sfdefault} 
\usepackage[T1]{fontenc}

 \linespread{1.3}

\begin{document}

\begin{small}

\begin{center}
\section*{Wyznaczanie przyspieszenia ziemskiego przy pomocy wahadła matematycznego}
\subsubsection*{Wstęp teoretyczny}
\end{center}
\subsection*{Siła grawitacji} 

\textit{\underline{Siła grawitacji}} to w typowym dla nas - ludzi rozumieniu przyciąganie otaczających nas przedmiotów przez glob ziemski, czyli zjawisko ciężaru. W fizyce jest to jedno z oddziaływań podstawowych, określone jako zjawisko naturalne, polegające na tym, że wszystkie obiekty , które posiadają masę oddziałują na siebie wzajemnie i przyciągają się. Kiedy znajdujemy się na powierzchni Ziemi, odległość od środka jej ciężkości jest dużo większa niż wysokość, na której możemy się przemieszczać. W takiej sytuacji można przyjąć, że pole grawitacyjne jest z dużą dokładnością jednorodne. 

Korzystając z zależności na siłę grawitacyjną można obliczyć, że na przedmiot o masie $m$ na powierzchni naszej planety działa siła $F_g$: 
\begin{Large}
$$F_g=\frac{GM_z m}{r_z^2}$$
\end{Large}
\begin{small}
gdzie: $M_z \approx 5,9736 \cdot 10^{24} kg$ - masa Ziemi, $r_z \approx 6373,14 km$, a zgodnie z drugą zasadą dynamiki:
\end{small}
\begin{Large}
$$a=\frac{F_g}{m}$$
\end{Large}
\begin{flushright}
\begin{scriptsize}
Źródło: \textit{http://www.fizykon.org/grawitacja/grawitacja$\_$wprowadzenie.htm} \\
\textit{http://pl.wikipedia.org/wiki/Grawitacja}
\end{scriptsize}
\end{flushright}
\subsection*{Przyspieszenie ziemskie, jednostka, zależność wartości od szerokości geograficznej i wysokości nad poziomem morza}
\textit{\underline{Przyspieszenie ziemskie}} - przyspieszenie grawitacyjne ciał swobodnie spadających na Ziemię, bez oporów ruchu.Podstawiając zależność na siłę można obliczyć przyspieszenie ziemskie $g$:
\begin{Large}
$$g=\frac{GM_z}{r^2}\approx\frac{6,6732 \cdot 10^{-11} \cdot m^3 \cdot kg^{-1} \cdot s{-2} \cdot 5,9736 \cdot 10^{24}kg}{(6373,14 km)^2} \approx 9,81 \frac{m}{s^2}$$
\end{Large}
Jednostkami przyspieszenia ziemskiego są jednostki przyspieszenia: 
$$[g]=[\gamma]=\frac{N}{kg}=\frac{m}{s^2}$$
Wartość przyspieszenia ziemskiego zależy od szerokości geograficznej oraz wysokości nad poziomem morza. Wraz z wysokością przyspieszenie maleje odwrotnie proporcjonalnie do kwadratu odległości do środka Ziemi i jest wynikiem zmniejszania się siły grawitacji zgodnie z prawem powszechnego ciążenia. Zmniejszanie się przyspieszenia ziemskiego wraz ze zmniejszaniem szerokości geograficznej jest spowodowane działaniem pozornej siły odśrodkowej, która powstaje na skutek ruchu obrotowego Ziemi. Ponieważ siła ta jest proporcjonalna do odległości od osi obrotu, stąd największą wartość osiąga na równiku. Ponieważ siła odśrodkowa ma tu zwrot przeciwny do siły grawitacji, przyspieszenie ziemskie na równiku osiąga najmniejszą wartość. Dodatkowe zmniejszenie przyspieszenia ziemskiego w okolicach równika spowodowane jest spłaszczeniem Ziemi (większą odległością od środka Ziemi).\\
Nie obserwuje się zależności przyspieszenia ziemskiego od długości geograficznej.\\
Przybliżoną zależność przyspieszenia ziemskiego, z uwzględnieniem podanych efektów, podaje wzór:
$$g_{\varphi}\approx 9,78318(1+0,0053024sin^2\varphi-0,0000058sin^22\varphi)-3,086\cdot 10^{-6}h$$
\begin{small}
gdzie: $h$ - wysokość nad poziomem morza [m], $\varphi$ - szerokość geograficzna [$^{\circ}$]
\end{small}
\begin{flushright}
\begin{scriptsize}
Źródło: \textit{http://pl.wikipedia.org/wiki/Przyspieszenie$\_$ziemskie}
\end{scriptsize}
\end{flushright}
\subsection*{Wahadło matematyczne} 
Idealne \textit{\underline{wahadło matematyczne}} to punktowa masa $m$ zawieszona na nieważkiej i nierozciągliwej lince. W rzeczywistości niewielka ciężka kulka zawieszona na mocnej lince i cienkiej nici jest dobrym przybliżeniem. 
\begin{flushright}
\begin{scriptsize}
Źródło: \textit{http://fizyka.pisz.pl/strona/121.html}
\end{scriptsize}
\end{flushright}

\subsection*{Zależność okresu drgań od długości wahadła matematycznego} 
Okres drgań wahadła matematycznego jest niezależny od masy $m$ i wynosi dla małych kątów $\alpha$
$$T=2\pi \sqrt{\frac{l}{g}}$$
gdzie, $l$ - długość wahadła
\begin{flushright}
\begin{scriptsize}
Źródło: \textit{http://fizyka.pisz.pl/strona/121.html}
\end{scriptsize}
\end{flushright}


\newpage
\subsection*{Przebieg i cel ćwiczenia}
	Celem tego ćwiczenia jest wyznaczenie przyspieszenia ziemskiego przy pomocy wahadła matematycznego. 
	Nasze stanowisko pomiarowe składa się z osadzonej w urządzeniu poprzeczki ze skalą milimetrową, na której jest zawieszone wahadło matematyczne. Długość wahadła można zmniejszać za pomocą pokrętła. Składa się również z czasomierza, który wykorzystuje złącze optoelektroniczne - fotokomórkę i mierzy czas $N$ wahnięć wahadła w sekundach (W naszym wypadku $N=10$). Na początku ćwiczenia ustalono początkową długość wahadła ($L=30 cm$). Fotokomórka umieszczona jest tak, aby kula zawieszona na sznurku, podczas wykonywania wahnięć przechodziła przez jej światło, tym samym uruchamiając ją. \\
	Ćwiczenie polegało na wykonaniu pięciokrotnie pomiarów czasu wahnięć $[s]$ przy danej długości wahadła, następnie należało zwiększać długość wahadła. Nasz zakres długości to $30-90 cm$. Czasomierz mierzył czas $10$ wahnięć. Odchylenie wahadła matematycznego nie mogło być większe niż $7^{\circ}$. Odchylenie wahadła było kontrolowane przez kątomierz umieszczony na górze.
\subsection*{Opracowanie pomiarów}
Pomiary zostały zapisane w tabeli: 
\begin{center}
\begin{tabular}{|c|c|c|c|c|c|} \hline
$L[cm]$ & \multicolumn{3}{c|}{$t[s]$}\\ \hline
 & 1 & 2 & 3 & 4 & 5 \\ \hline
30 & 11,02 & 11,03 & 11,02 & 11,03 & 11,01 \\ \hline
35 & 11,92 & 11,94 & 11,94 & 11,95 & 11,94 \\ \hline
40 & 12,75 & 12,75 & 12,75 & 12,77 & 12,76\\ \hline
45 & 13,52 & 13,52 & 13,51 & 31,53 & 13,51\\ \hline
50 & 14,25 & 14,25 & 14,26 & 14,27 & 14,28 \\ \hline
55 & 14,99 & 14,97 & 14,98 & 14,98 & 14,98 \\ \hline
60 & 15,28 & 15,27 & 15,28 & 15,28 & 15,28\\ \hline
65 & 16,25 & 16,26 & 16,26 & 16,26 & 16,27\\ \hline
70 & 16,88 & 16,88 & 16,87 & 16,88 & 16,87 \\ \hline
75 & 17,43 & 17,43 & 17,43 & 17,43 & 17,43\\ \hline
80 & 17,99 & 17,99 & 17,99 & 17,99 & 18\\ \hline
85 & 18,53 & 18,54 & 18,54 & 18,55 & 18,55 \\ \hline
90 & 19,06 & 19,06 & 19,06 & 19,07 & 19,06\\ \hline
\end{tabular}

\end{center}
Niepewności pomiarowe wynoszą odpowiednio:\\ - czasomierz: $\Delta t=0,01 s$ \\- skala milimetrowa: $u(L)= 0,5 cm$
\newpage
Dla każdej długości wahadła obliczono wartości $\sqrt{L}$ oraz średnie wartości mierzonego czasu $N$ wahnięć:
\begin{center}
\begin{tabular}{|c|c|} \hline
$t_{sr}$ & $\sqrt{L}$ \\ \hline
11,022 & 0,547722558 \\ \hline
11,936 & 0,591607978 \\ \hline
12,756 & 0,632455532 \\ \hline
13,518 & 0,670820393 \\ \hline
14,262 & 0,707106781 \\ \hline
14,98 & 0,741619849 \\ \hline
15,278 & 0,774596669  \\ \hline
16,26 & 0,806225775 \\ \hline
16,876 & 0,836660027\\ \hline
17,43 & 0,866025404 \\ \hline
17,992 & 0,894427191\\ \hline
18,542 & 0,921954446 \\ \hline
19,06 & 0,948683298\\ \hline

\end{tabular}
\end{center}
Następnie została obliczona statystyczna niepewność typu $u_a(t_{sr})$, dla każdej długości wahadła,  jako odchylenie standardowe wartości średniej, pomnożone przed odpowiedni współczynnik Studenta 
Fishera, który w tym przypadku wynosił $1,141$. Otrzymano następujące wyniki:
\begin{center}
\begin{tabular}{|c|c|} \hline
Lp. & $u_a(t_{sr})$ \\ \hline
1. & 0,009546291 \\ \hline
2. & 0,013009402 \\ \hline
3. & 0,010205414\\ \hline
4. & 0,009546291\\ \hline
5. & 0,01487682\\ \hline
6. & 0,008068088\\ \hline
7. & 0,005102707\\ \hline
8. & 0,008068088\\ \hline
9. & 0,006249514\\ \hline
10. & 0\\ \hline
11. & 0,005102707\\ \hline
12. & 0,009546291\\ \hline
13. & 0,008068088\\ \hline
\end{tabular}

\end{center}
Korzystając ze wzoru ogólnego $$u(x) = \frac{\Delta_{p}x}{\sqrt{3}}$$ obliczono $u_b(t)$. Wzór wygląda następująco: $$ u_b(t)=\frac{\Delta t}{\sqrt{3}}$$, a wartość $u_b(t)$ wynosi: \textbf{ 0,005773503}. \\
Niepewności pomiarowe całkowite średnich czasów wyliczone ze wzoru: $$u(t_{sr})=\sqrt{u_a^2(t_{sr})+u_b^2(t)} $$ wynoszą:
\begin{center}

\begin{tabular}{|c|} \hline
 $u_a(t_{sr})$ \\ \hline
0,011156387 \\ \hline
0,014232984 \\ \hline
0,011725348 \\ \hline
0,011156387 \\ \hline
0,015957853 \\ \hline
0,009921056 \\ \hline
0,007705253 \\ \hline
0,009921056 \\ \hline
0,008508216 \\ \hline
0,0057735  \\ \hline
0,007705253 \\ \hline
0,011156387 \\ \hline
0,009921056\\ \hline
\end{tabular}

\end{center}
Następnie obliczono okres drgań $T$ ze wzoru: 
$$ T=\frac{t_{sr}}{N} $$ 
,wiedząc, że $N=10$
\begin{center}
\begin{tabular}{|c|} \hline
 $T$ \\ \hline
1,1022\\ \hline
1,1936\\ \hline
1,2756 \\ \hline
1,3518 \\ \hline
1,4262 \\ \hline
1,498 \\ \hline
1,5278 \\ \hline
1,626  \\ \hline
1,6876  \\ \hline
1,743 \\ \hline
1,7992 \\ \hline
1,8542 \\ \hline
1,906\\ \hline
\end{tabular}
\end{center}
Korzystając z prawa propagacji niepewności obliczono niepewności powyższych danych ze wzoru: $$ u(T)=|\frac{u(t_{sr})}{N}|$$ i wynoszą one:
\begin{center}
\begin{tabular}{|c|} \hline
 $u(T)$ \\ \hline
0,001115639\\ \hline
0,001423298\\ \hline
0,001172535 \\ \hline
0,001115639 \\ \hline
0,001595785\\ \hline
0,000992106\\ \hline
0,000770525 \\ \hline
0,000992106  \\ \hline
0,000850822  \\ \hline
0,00057735 \\ \hline
0,000770525 \\ \hline
0,001115639 \\ \hline
0,000992106\\ \hline
\end{tabular}
\end{center}
Na podstawie wszystkich danych sporządzono wykresy: T(N) i T($\sqrt{N}$). Na obu wykres niepewności nie zostały zaznaczone, gdyż są one zbyt małe i nie byłyby widoczne na wykresie.
\begin{flushleft}


\includegraphics[scale=0.8]{zad1}
\end{flushleft}

\begin{flushleft}


\includegraphics[scale=0.68]{zad2}
\end{flushleft}

Na powyższym wykresie dopasowano prostą za pomocą regresji liniowej. Równanie prostej wynosi: $y = 2,0019x + 0,0071$ oraz $R^2 = 0,9987$. \\ Niepewności współczynników wynoszą kolejno: $u(a) = 0,02164, u(b) = 0,01676$

Na podstawie wyliczonego współczynnika, można obliczyć przyspieszenie ziemskie, korzystając ze wzoru opartego na równaniu ruchu: $$g=\frac{4\pi^2}{a^2}$$, co po podstawieniu dało wynik: \textbf{ g = 9,850590064} [$\frac{m}{s^2}$]. \\
Następnie korzystając z prawa niepewności można obliczyć niepewność $u(g)$ ze wzoru: $$u(g)=\sqrt{{(\frac{4\pi^2}{a^2}\cdot u(a))^2}}$$, co dało wynik: \textbf{u(g)=0,004614123}$[\frac{m}{s^2}]$
Niepewność rozszerzona została obliczona ze wzoru $$u_p(g)=k_p\cdot u(g),k_p=2$$ i wynosi \textbf{$u_p(g)=$ 0,009228247}
Następnie obliczono wartości przyspieszenia ziemskiego dla szerokości geograficznej i wysokości nad poziomem morza dla Gliwic. Do obliczenia były nam potrzebne wartości:\\$h=240[m.n.p.m.] $ \\ $\varphi =50,31 [^\circ]$ Następnie podstawiając do wzoru:
$$g_\varphi\approx 9780318(1+0,0053024sin^2\varphi-0,0000058sin^2 2\varphi)-3,086 \cdot10^{-6}h$$ otrzymujemy wynik dla Gliwic:  \\\textbf{g=9,7796} $[\frac{m}{s^2}]$

\subsection*{Test zgodności}
Dla otrzymanej wartości test zgodności został przeprowadzony na podstawie następującego wzoru: $$\frac{|x-y|}{\sqrt{u^2(x)+u^2(y)}}$$ i wynosi 5,31 co jest większe od 3.	
\subsection*{Wnioski}
 Obliczona wartość przyspieszenia ziemskiego nie jest zgodna z wartością tablicową, nawet dla potrojonej niepewności $u(g)$. Przyczyną niedokładnego wyniku może być zróżnicowanie odchylenia wahadła matematycznego od pionu, mimo, że pomiary przy tych samych długościach wahadła były niemal identyczne.
\end{small}
\end{document} 
